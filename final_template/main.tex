%  A simple AAU report template.
%  2015-05-08 v. 1.2.0
%  Copyright 2010-2015 by Jesper Kjær Nielsen <jkn@es.aau.dk>
%
%  This is free software: you can redistribute it and/or modify
%  it under the terms of the GNU General Public License as published by
%  the Free Software Foundation, either version 3 of the License, or
%  (at your option) any later version.
%
%  This is distributed in the hope that it will be useful,
%  but WITHOUT ANY WARRANTY; without even the implied warranty of
%  MERCHANTABILITY or FITNESS FOR A PARTICULAR PURPOSE.  See the
%  GNU General Public License for more details.
%
%  You can find the GNU General Public License at <http://www.gnu.org/licenses/>.
%
\input{setup/preamble.tex}% package inclusion and set up of the document

\input{setup/hyphenations.tex}% 
%Character count macro
\newread\tmp

\newcommand{\quickcharcount}[1]{%
  \immediate\write18{texcount -1 -sum -merge -char #1.tex > #1-chars.sum}%
  \openin\tmp=#1-chars.sum%
  \read\tmp to \thechar%
  \closein\tmp%
}

\newcommand{\quickwordcount}[1]{%
  \immediate\write18{texcount -1 -sum -merge #1.tex > #1-words.sum}%
  \openin\tmp=#1-words.sum%
  \read\tmp to \theword%
  \closein\tmp%
}

% my new macros

\begin{document}
%frontmatter
\pagestyle{empty} %disable headers and footers
\pagenumbering{roman} %use roman page numbering in the frontmatter
%  A simple AAU report template.
%  2015-05-08 v. 1.2.0
%  Copyright 2010-2015 by Jesper Kjær Nielsen <jkn@es.aau.dk>
%
%  This is free software: you can redistribute it and/or modify
%  it under the terms of the GNU General Public License as published by
%  the Free Software Foundation, either version 3 of the License, or
%  (at your option) any later version.
%
%  This is distributed in the hope that it will be useful,
%  but WITHOUT ANY WARRANTY; without even the implied warranty of
%  MERCHANTABILITY or FITNESS FOR A PARTICULAR PURPOSE.  See the
%  GNU General Public License for more details.
%
%  You can find the GNU General Public License at <http://www.gnu.org/licenses/>.
%
\pdfbookmark[0]{Front page}{label:frontpage}%
\begin{titlepage}
  \addtolength{\hoffset}{0.5\evensidemargin-0.5\oddsidemargin} %set equal margins on the frontpage - remove this line if you want default margins
  \noindent%
  \begin{tabular}{@{}p{\textwidth}@{}}
    \toprule[2pt]
    \midrule
    \vspace{0.2cm}
    \begin{center}
    \Huge{\textbf{
      Exploring the impact of language differences on GANs for Password Cracking % insert your title here
    }}
    \end{center}
    \begin{center}
      \Large{
        
        %Subtitle % insert your subtitle here
      }
    \end{center}
    \vspace{0.2cm}\\
    \midrule
    \toprule[2pt]
  \end{tabular}
  \vspace{2 cm}
 
\begin{center}
	\includegraphics[scale = 0.75]{figures/ruc_real_logo.png}\\[1.0 cm]	% University Logo
	\textsc{\LARGE\bfseries Roskilde University}\\[2.0 cm]	% University Namecontent...
	\vspace{-1.9cm}
	\textsc{\Large  Computer Science Department}\\[0.5 cm]	
\end{center}
\vspace{1.5cm}
 \begin{flushleft}
	{\large\bfseries
		Author%Insert document type (e.g., Project Report)
	}\\
	\vspace{0.2cm}
	{\Large
		Eugenio Maria Capuani%Insert your group name or real names here
	}
	
\end{flushleft}
\begin{flushright}
	\vspace{-1.6cm}
	{\large\bfseries
		Supervisor%Insert document type (e.g., Project Report)
	}\\
	\vspace{0.2cm}
	{\Large
		Niels Jørgensen %Insert your group name or real names here
	}
	
\end{flushright}
\vspace{1cm}
  \vfill
  \begin{center}
  \today
  \end{center}
\end{titlepage}

%\pdfbookmark[0]{English title page}{label:titlepage_en}
\aautitlepage{%
  \englishprojectinfo{
    Exploring the impact of language differences on GANs for Password Cracking   %title
  }{%
    IT Security, Artificial Intelligence %theme
  }{%
    Spring Semester 2019 %project period
  }{%
    37 % Number of Normal pages
  }{%
    %list of group members
    %Eugenio Maria Capuani\\ 
    %Kidane Mahari Tesfai\\
    Eugenio Maria Capunai%\\
    %Rune Bastian Barrett
  }{%
    %list of supervisors
    Niels Jørgensen
  }{%
    90065 % number of characters
  }{%
    \today % date of completion
  }%
}{%department and address
  \textbf{Computer Science Department}\\
  Roskilde University\\
  \href{http://www.ruc.dk}{http://www.ruc.dk}
}{ %the abstract
The goal of this thesis is to explore what impact language has on password cracking (if any) when using Generative Adversarial Networks (GANs) to crack passwords.
In it we build on existing research by taking an existing GAN-based password cracker, and testing it with a dataset of leaked user passwords from Italy. 
By comparing the performance of the resulting model with widely-used wordlists based primarily on English-language sources, as well as state-of-the-art rule-based tools, we hope to get insights into the impact that different grammatical structures have on both the performance of the GAN model, and by extension on its performance as a password cracking tool.
%\newline
In our testing we were able to crack almost 80\% of the Italian passwords in our database using GANs, while other state-of-the art tools only managed 71\%. Furthermore, we recovered a number of unique language-specific passwords that traditional tools would not have been able to find. 

We also explored the impact of training GANs with natural language data in addition to password data, but found performance was about the same as without the extra natural language data.

We conclude that language differences do have an impact on GAN-based tools, and that these tools are expressive enough to adapt to such differences. Overall GANs can be an effective tool for attacking non-english passwords, both in addition to state-of-the art tools or by themselves.
}

% \cleardoublepage
% {\selectlanguage{danish}
% \pdfbookmark[0]{Danish title page}{label:titlepage_da}
% \aautitlepage{%
%   \danishprojectinfo{
%     Rapportens titel %title
%   }{%
%     Semestertema %theme
%   }{%
%     Efterårssemestret 2010 %project period
%   }{%
%     XXX % project group
%   }{%
%     %list of group members
%     Forfatter 1\\ 
%     Forfatter 2\\
%     Forfatter 3
%   }{%
%     %list of supervisors
%     Vejleder 1\\
%     Vejleder 2
%   }{%
%     1 % number of printed copies
%   }{%
%     \today % date of completion
%   }%
% }{%department and address
%   \textbf{Elektronik og IT}\\
%   Aalborg Universitet\\
%   \href{http://www.aau.dk}{http://www.aau.dk}
% }{% the abstract
%   Her er resuméet
% }}

\cleardoublepage
\pdfbookmark[0]{Contents}{label:contents}
\thispagestyle{empty} %enable headers and footers again
\tableofcontents
\pagestyle{fancy}
%\listoftodos

%\input{sections/preface.tex}
\cleardoublepage
%mainmatter
\pagenumbering{arabic} %use arabic page numbering in the mainmatter
\begin{abstract}
	\thispagestyle{plain}
	\setlength{\parskip}{5pt}
	\setlength{\parindent}{0pt}
	
	The goal of this thesis is to explore what impact language has on password cracking (if any) when using Generative Adversarial Networks (GANs) to crack passwords.
	In it we build on existing research by taking an existing GAN-based password cracker, and testing it with a dataset of leaked user passwords from Italy. 
	By comparing the performance of the resulting model with widely-used wordlists based primarily on English-language sources, as well as state-of-the-art rule-based tools, we hope to get insights into the impact that different grammatical structures have on both the performance of the GAN model, and by extension on its performance as a password cracking tool.
	
	To that end we also wish to explore whether including Italian language corpora during training is decremental to the network's performance (as some prior research seems to indicate), or if it can prove beneficial in this particular context. 
\end{abstract}
\clearpage
%ADD A MORE GENERAL QUESTION TO THE PROBLEM FORMULATION, THE $ EXISITING ONES WOULD BE SUB-QUESTIONS

\section{Introduction}\label{sec:introduction}
\subsection{Motivation}
The aim of this thesis is to evaluate the paper \enquote{PassGAN: A Deep Learning Approach for Password Guessing}\cite{PassGAN}, by testing the Deep Learning system described in the paper with a different dataset of leaked passwords from Italy\cite{libero_leak} (referred to as the Libero dataset in this thesis). The aim of the thesis is to test whether there are any differences in performance when PassGAN is used to crack a database from a non-english source.

This dataset is composed of real user passwords belonging to the Italian email provider Libero Mail, that were leaked in 2016. %We believe the use of these passwords to be ethical because this particular leak has been public for a number of years. %And the company has since taken action to remediate?

The underlying thought behind this it to test whether differences in grammar and language have any noticeable effect when training the system: perhaps GANs are expressive enough to account for the different provenance of the data without any significant change, or maybe some adjustments should be made such as the inclusion of Natural Language corpora in the training data. Another possible change might be to mix password datasets from English and non-English sources.

The inclusion Natural language corpora might present its own challenges, such as the fact that some prior studies with Recurrent Neural Networks indicate that the inclusion of such data ends up creating a lot of noise rather than improving performance\cite{Melicher2016}.

One might make the case that linguistic differences are not that relevant when it comes to passwords, that the use of grammatical constructs is often trumped by patters of user behaviour in password creation that are international and well represented in rule-based password crackers. Thus, it should not matter where the network learns these patterns.\\
On the other hand, in our early attempts to train PassGAN on the Libero dataset we observed that most of the passwords the system generates attempt to mimic the sound and structure of Italian words; this leads us to speculate that the inclusion on natural language corpora might help the system to generate grammatically correct words, and thus perhaps improve performance. %it looks like its trying to mimic italian

In conclusion, we believe this research might contribute some insight in the role that grammatical features have in password cracking, when this task is approached using Deep Neural Networks: many papers on the subject (including \cite{PassGAN} and \cite{Melicher2016}) ask the question of whether Deep Learning systems are expressive enough to generate novel passwords and thus yield results that are not achievable with traditional password cracking methods, and we see this research as part of this ongoing quest into exploring the capabilities and limits of Deep Learning-based password crackers.

\subsection{Problem formulation}\label{subsec:problem-formulation}
This Thesis aims to answer the following questions:
\begin{itemize}
\item How does PassGAN perform when cracking italian passwords? %Implies: is it better or worse when using the pre-trained model as opposed to a new model trained on italian passwords?
\item Does difference in language have an impact on the passwords found by PassGAN and state-of-the-art password crackers?    
\item How does the inclusion of natural language data during training affect PassGAN's performance?
\item Ultimately, can PassGAN be a useful tool to use when approaching password data from a particular language area? %Or are rule-based tools always better    
\end{itemize}


\cleardoublepage
%Talk more about Thompson and morris, they introduce all the major techiques like hashing and salting at re-iterating the hash function.

\section{Related Work}
Both Password Cracking and deep learning are active areas of research developing at a rapid pace.

In this section, we aim to give an overview of the relevant knowledge in these areas as it relates to our thesis.

\subsection{Password Cracking}
Password cracking has been around for a long time, and while technology has evolved greatly and continues to do so, the basic concepts have remained relatively similar:
Back in 1979, Morris and Thompson were already exploring different ways to attack passwords and defend against such attacks \cite{Thompson1979}.

At the base of all password attacks is the concept of \emph{key space}, i.e the set of all possible passwords of a certain length that use a specific character set \cite{Thompson1979,hash_cat_mask_attack}. 
%Key sooace is theoretically infinite. add that.
%Maybe you could use that paper full of set thory math. Just check the scholarship beforefand.
In the case of the password \texttt{password}, the key space can be expressed $26^8$ (i.e The number of possible characters to the length of the string).

Key space is important because it is the basis from which password cracking techniques work off: if the password is sufficiently complex, an attacker will have to do an exhaustive search of the search space in order to find the password. 
It follows then, that one of principles of password strength is to make the search space big enough to be impractical to search through with current hardware.
This is the reason why users are commonly advised to use a variety of different characters classes in their passwords.

To ground our discussion of search space in the context of cracking, we should address how passwords are stored in the first place: As Morris and Thompson explain in theirr paper, the simplest approach is to store the users' passwords ina file or databse as they are entered: this is a bad idea as any softwre bug that causes an accidental disclosure will leave the users exposed, and also because any priviledged user can simply look up the other users' passwords.

A better approach would be to encrypt the user's password, and store the cypher text: when a user logs in, the sring they typed is compared with the cypher text and access is granted if it matches. This encryption is commonly acheived through a one-way criptographic hash function, and Morris and Thompson use the DES algorythm in their paper. %Check that the terminology is correct.
Simply put, a cryptographic hash function is a mathematical function that, given a seed or key, encrypts a string in a predictable way; another feature of such function is that converting cypher text back into plain text should be exeedingly difficult, thus making reversing the process unfeasable.
This propriety however changes as the amount of available computing power increases, and any given hashing algorithm will eventually become insecure.

When a strong hashin algorithm is used, an attacker will theoretically have do do a key space search in order to crack the users' passowrd: this process will take even more time, as every candidate string must be first encrypted with the same algorithm and then looked up in the database.

Encryption alone does not solve the problem however, because in reality an attacker would not have to resort to brute-forcing in a majority of cases.

Bacause the hashing alorithm needs to always output the same result for a given input string, the attacker can start comparing the hashes in the database and draw some conclusion: if some hashed appear many times, they probably hold the plain text of some very common passwords; If the attacker cracks those first and starts looking for similar patterns, he may crack a sizable number of the passwords contained in the database.

This leads us to our next attack strategy, which is to build a table with Pre-computed hashes for all the most common passwords: this technique saves a lot of time since we do not need to encrypt every candidate password before comparing it with the database. These tables are commonly referred to as \emph{Rainbow Tables}, and they are a simplified version of the rule-based techniques covered in section \ref{hash_and_jtr}.

Rainbow tables can be defeated by using \emph{password salting}: salting works by generating a random string that is then appended to the user's password before it is encrypted

%Finish talking about salting
%Use NIST Page for modern equivalnts   

An attacker's goal is to try to shrink the subset of the search space that needs to be examined in order to speed up the process, many times exploiting user habits or inherit characteristics of the password.\newline


Password cracking can be accomplished in many ways, and the approach often depends on the situation: the attacker might be trying to gain access to a system by cracking the password of a high-privilege user like a System Administrator, or he might be cracking a leaked password database in order to later sell the personal data contained within.

In the first scenario, we might try to learn as much as we can about the target and try to use social engineering techniques in order to obtain the password; Phishing and various kind of fraud are commonly used in such cases. %Needs a reference if you wanna keep it!

In the second scenario, we are trying to extract as many password as possible from the leaked database and we might take advantage of user behavior in order to do so: Users tend to choose common passwords and to re-use the same password in multiple instances.%Also needs a reference 
If we were to attack common passwords first, we can significantly reduce the combine search space.
%Add more about passwords proper, for example about hashing algorithms and salting?

\subsection{Rule-Based password crackers} \label{hash_and_jtr}
%Define rules firrst in the introduction

Two common tools for password cracking are John The Ripper and \break \mbox{HashCat}\cite{john,hash_cat}: these tools take advantage of common patterns in user behaviour in order to optimize the cracking process: Both tools have a variety of techniques an attacker can employ, and we will briefly exemplify their capabilities using HashCat as an example \cite{hash_cat_wiki}: \newline

In both tools there are three categories of attacks that can be carried out: brute-force attacks, dictionaries attacks and rule based attacks: these can be combined and tweaked in various ways depending on the desired result, and there is a degree of overlap between each.\newline 
%Devidide it in 3 sub-sections and define them better

HashCat's \enquote{modes} reflect these categories, and the simplest of these is \emph{Mask attack} mode: this is modified version of key space search, meant to attack simpler passwords while shrinking the key space search: 
instead of searching the total space for a password of length $x$, we define a simple regular pattern expressing the what character classes are there and at which position, saving us a substantial amount of processing time.
For example is we have a password like \texttt{Benjamin86} or \texttt{Iloveyou02}, we can define our pattern to be a ten-character string with eight lowercase or uppercase letters and two numbers at the end; This is referred to as a \emph{Mask} in HashCat's documentation:

In a classic brute-force attack we would deal with a search space of $62^{10}$ (or roughly $8 \times 10^{17}$ combinations), but thanks to the above-mentioned mask we can reduce our search space to around $4 \times 10^{13}$ possibilities if we assume that the first character in the string is the only one that can be uppercase.
Furthermore we can use the \texttt{--increment } option in HashCat to apply this pattern to all strings up to ten characters, allowing us to match shorter passwords that follow the same pattern.

This method is rather simplistic and not very flexible, but exemplifies some of the ways in which attackers can optimize key space attacks.\newline

The main mode of operation oh HashCat \emph{straight} mode, that performs a dictionary attack: In this mode, the program is fed a wordlist/dictionary, and tries each entry the wordlist as a password candidate. Because of its simplicity, such a dictionary attack works best with a wordlist composed of leaked passwords; the aim is to target very common passwords and users that re-use passwords, but the effectiveness of such an attack can increase significantly depending on what wordlist is used.

Dictionary attacks can be further enhanced by combining them in various ways: one approach would be to use two wordlists and append/prepend each entry in the second one to each entry in the first; 
the second wordlist might be a natural language dictionary or a wordlist of plain-text leaked passwords. This is called \emph{Combinator attack} mode in HashCat. 
We might also want to use the output of a brute-force attempt as out second wordlist: If we use the patterns described in the Mask Attack mode to generate strings we combine with a wordlist, we will obtained a more targeted and effective version of the Mask Attack method. 
For example is we know that a good deal of the passwords we want to extract are strings with numbers appended to the end, we might run a mask that generates combinations of 0 to 4 digits and then combine the output of that with our wordlist. This is called \emph{Hybrid attack } mode in HashCat.\newline

Finally we come to rule-based attacks: In short they are an extension of all the methods described above, and all the aforementioned techniques can also be performed using rules; however, rules are more flexible and allow for more thorough definition of the patterns that may appear in a password, going beyond the capabilities of a regular language.
Patterns can be created independently of the size and characteristics of the passwords, and they are not limited to a fixed patterns; there are also flow control statements and options to apply rules only in certain conditions.
There are also options to save password candidates to memory enabling more advanced processing: saved strings can be appended to each pasword candidate matching certain criteria, reversed and so on...

Rules are applied to each entry in a wordlist in a similar way to a combinator attack, and multiple rules can also be be applied sequentially to the same entry in the wordlist.
Rules provide a more efficient way to tackle password cracking since their greater flexibility means that an attacker need not know as much about their target. 

%ADD SOME RULES EXAMPLES

 
%They are fed a list of strings to search in the database, and a set of mangling rules: they first compare the list with the database, and then go through the process again but applying each rule to each entry in the word-list; In order to crack passwords more efficiently, words-lists usually contain lists of common passwords or real passwords from other data leaks, but they may also contain dictionary words and natural language fragments.

%The rules are given in a regular language and express common things that users do when choosing passwords (such as adding numbers at the end of a string, toggling the case of the first letter or perform common pattern substitutions like \enquote{leet speak}). because of these mangling rules, a password like \texttt{P4ssw0rd1} may seem strong, but it will be cracked quite effortlessly.

\clearpage
\section{Deep Learning}

Deep Learning is a class of machine learning systems whose goal is to extract relevant features from a distribution of data. On an abstract level, Deep Learning systems take inspiration the structure of human biology, with multi-layer structures of nodes (Neurons): each layer might be thought of as a section of a brain recognizing a very specific element, and when all layers work in unison they can recognize and act upon high level features and categories.

Deep learning is used in many fields, and has achieved notable results in fields such as Computer Vision, Image Recognition and Natural Language Processing.%REFERENCE

Deep Learning Systems are particularly useful because of their ability to extract high level features from data, especially features that are hard to define algorithmically by humans.

Another peculiar features of such network is that no-one really knows exactly how the machine learns. There has been such research trying to understand what exactly each layer does and what exactly it learns, but this is still an open question.\newline %REFERENCE.
Because of the number of variables involved, it is hard to predict results might derive from a change in the initial condition; as a consequence of this uncertainty, the process of working with Deep Learning system is one of incremental change and experimentation. 

Figure \ref{fig:DNN} shows the typical structure of a Deep Neural Network:
\begin{figure}[H]
\centering    
\begin{neuralnetwork}[height=8]
\newcommand{\x}[2]{$x_#2$}
\newcommand{\y}[2]{$y_#2$}
\newcommand{\hfirst}[2]{\small $h1_#2$}
\newcommand{\hsecond}[2]{\small $h2_#2$}
\inputlayer[count=4, bias=false, title=Input Layer, text=\x]
\hiddenlayer[count=8, bias=false, title=Hidden Layer 1, text=\hfirst] \linklayers
\hiddenlayer[count=8, bias=false, title=Hidden Layer 2, text=\hsecond] \linklayers
\outputlayer[count=4, title=Output Layer, text=\y] \linklayers
\end{neuralnetwork}
\caption{An example of a Deep Neural Network}\label{fig:DNN}
\end{figure}
As we can see in the example above, a neural network is composed of an Input Layer, a number of Hidden layers and an Output layers.
Speaking in general terms, the input layer receives the chunk of data to be processed, the hidden layers operate on the input sequentially, and the output layer is where the the system expresses what it beleives the data to mean.

Training of a neural network is composed of two distinct phases: the \emph{Feed Forward} phase and the \emph{Back-Propagation} Pahse: In the feed forward phase the network processes the current data and comes to a result, then the networks computes the Error (or how much the result it had come to deviated from the expected results), and tweaks its parameters in an effort to reduce the error.
Once both phases are completed the network has completed an iteration. The next bacth of data is then introduced and the process begins anew.

In this coming section we will cover two types of Deep Learning Neural Networks that are relevant in this paper: \emph{Recurrent Neural Networks} (RNNs) and  \emph{Generative Adversarial Networks} (GANs).
\clearpage

\subsection{Recurrent Neural Networks}
Recurrent Neural Networks are the simpler of the two network architectures; they are a kind of neural network specialized in working with data that has a temporal component.
For instance if the network needs to predict the next letter or word in a sentence, it needs to know what came before. Another example might be a neural network that scales or edits videos, where the action to take on any given frame might be dependent on the frames that came before.

The precise method the network uses to keep track of temporal elements is rather complex, but can be briefly summarized by saying that each neuron in the network holds a state, and that each iteration a neuron's state from the previous iteration is fed as input to itself in the current iteration along side the current data to be processed.
This creates a feedback loop, whereby at any given point the calculation performed by each neuron are influenced by previous events.

Figure \ref{fig:rnn_neuron} explains more clearly.
\begin{figure}[H]    
\centering
\begin{tikzpicture}[shorten >= 1pt, node distance = 2cm, on grid, auto, square/.style={regular polygon, regular polygon sides = 4}]
\node[](0){};
\node[below of = 0](1){};
\node[below of = 1] (2){};

\node[state, right of = 1](N1){$H1_{(t)}$};
\node[state, right of = N1](N2){$H2_{(t)}$};
\node[state, right of = N2](output){$O$};

\draw[-]   (0) edge node{$x_i$} (N1)
            (2) edge node{$x_i$}(N1)
            (N1) edge (N2)
            (N2) edge (output);
\end{tikzpicture}
\begin{center}
Iteration number 1 $t=1$
\end{center}
\end{figure}

\begin{figure}[H]
\centering    
\begin{tikzpicture}[shorten >= 1pt, node distance = 2cm, on grid, auto, square/.style={regular polygon, regular polygon sides = 4}]
	\node[](0){};
	\node[below of = 0](1){};
	\node[below of = 1] (2){};
	
	\node[state, right of = 1](N1){$H1_{(t)}$};
	\node[state, right of = N1](N2){$H2_{(t)}$};
	\node[state, right of = N2](output){$O$};
	
	\draw[-]   (0) edge node{$x_i$}(N1)
	(2) edge node{$x_i$} (N1)
	(N1) edge[loop above] node{$H1_{t-1}$} (N1)
	(N2) edge[loop above] node{$H2_{t-1}$} (N2)
	(N1) edge (N2)
	(N2) edge (output);	
\end{tikzpicture}
\begin{center}
Iteration number 2 $t=2$
\end{center}
    \caption{A simplified representation of a node in a Reoccurrent Neural Network}\label{fig:rnn_neuron}
    
\end{figure}

% LSTM might be waaay too much detail. Also fuck its a *masive over-simplificatin*

For some particular tasks such as Natural language generation this setup is not enough, since as time goes on the network will develop a bias towards features present in more recent iteration; the influence of data from the older iteration will degrade over time, and the system will slowly tend to 'forget' features and patterns it learned at the beginnning: when it is then presented with new data it has never seen before, thi bias can pose a problem.

To dampen the effects of this bias one might employ Long-Short Term Memory cells (LSTM), a special kind on RNN neuron which can dynamically rank new information (and by extention, the content of its state) based on its relevance. LSTM ensures that important data is retained regardless on when it was learned, and thus diminishes the effect of the base RNN bias.\newline
It should be noted however that LSTM does not constitute a straight upgrade from the plain RNN architecture: it may be beneficial in certain cases, but decremental in others.
%maybe add a picture of a RNN or normal network for context?

\section{Generative Adversarial Networks} 

\clearpage
\section{Data Concerns with the libero set}
In the next section we will cover our experiments with PassGAN and the Libero dataset.
before that however we need to address some fundamental issues with the dataset: First, our copy of the Libero database onnly contained plain texts passowrds, and secondly we were unable to verify the exact provenance of this data.
A direct link to the data can be found on the \texttt{databases.today} website\footnote{\url{https://cdn.databases.today/Libero.it\%20900k.zip}}

As regards the first issue, as mentioned in section \ref{sec:related_work} the Libero set is a JSON formatted file containing various pieces of information for each of user (above 600,000 user); common fields for each user include email addrress, user-name and internal User ID, but some users in the document also have a real name attached. As for the passowrds, each user object contains both a plaintext password and an MD5 hash, but upon closer inspection we find that there is a problem: the MD5 hash is the same for each user, an MD5 hash of the word \enquote{boomerang}.

We have no idea as to why the uploader of the original cracker would choose to do something like that, and no clear evidence towards any particular reason; The following is purely our speculation, but we can imagine for example, that the attacker might decide to throw away the original hash and only keep the paintext of the password in order to keep the information for each user organized and more easily exploitable: since he exfiltrated not just password but also personal information like names and addresses for some of the users, one might imagine that it would be convenient to organize all the information for each user in a single JSON object.

This alteration to what we might normally expected to find - a file containing both plain texts and hashes - made us question the autenticity of the data: while we were unable to find definitive proof, the number of acounts present in our file is roughtly consisstent with the number reported in \cite{libero_leak}; further more, the hash of the archive matches that of a VirusTotal report, that shows the archive first appeared on VirusTotal in December 2016, a couple of months after news broke about the Libero Mail security breach \cite{virus_total}.
Italian news artices reporting on the incident came out around September  7th 2016, when Libero Mail sent an email to their customers informing them of the breach \cite{libero-news-wired,libero-news-tomhw,libero-news-fanpage}.
Looking at the database file itself, its last modified timestamp is dated to the 25th of September, placing its origin somewhere closer to the date of the incident: Unfortunately neither the news articles nor Libero Mail themselves give an indication of when exactly the incident occourred, so its hard to judge the exact timeline of events; Upon reading the email that Libero Mail sent to customers on September 7th (available in \cite{libero-news-fanpage} in Italian), the language use suggests they may have been aware of the breach for a considerable amount of time before breaking the news to customers.

The elements above leads to beleive that the data we have is geninuinely from the Libero leak, though ultimately its very hard to know for certain: these acounts and passwords have been included in a variety of other pastes ever since the incident, and if the original leak came from the Tor network like the Virus Total report tags seem to suggest, it might be inpossible to track down the original archive and/or creator.

   
%This fact however calls into question the autenticity of the data in our possession: Why have we used MD5 hashes in our experiments? and furthermore, how can we say that this data really belongs to the Libero.it leak that we claim?

%We have no definitive answer for either question, but we have some circumstantial evidence


\clearpage
\section{Testing and evaluation}

%TODO:  * rewrite the 10c thing and introcude it at the end the sampling section
%       * Re-organized all tasting explanation in the testing section, with a general introduction to it based on the diagram.
%       * Delete table about training set and remove all mentions of it.
%       * re-factor NL section

%THE TRAINIG SET IS NOT REALLY RELEVANT TO THIS CHAPTER, CONSIDER REMOVING MENTIONS OF IT AND ONLY SHOW TESTS AGAINST THE TESTING SET.
In this next section, we are going to evaluate the performance of the PassGAN system, as well as give a brief overview of our methodology and the experimental setup.

\subsection{Experimental setup}
Both training of the GAN and testing were carried out on a Linux machine running Debian stretch, equipped with a GTX 1070 graphics card, an i5-2500 CPU and 8 GB of system memory. 
With this card, training of PassGAN took roughly twelve hours.
PassGAN was run on Python 2.7, running Tensorflow 1.12.0.%\\

For testing we used the latest stable version of HashCat as of this writing (version 5.1.0). %and we observed a peak speed between 850 and 860MH/s (for reference, 1MH/s is equivalent to 1 million candidate passwords hashed and compared per second) %Reference  %The definition of MH is probably wrong

Figure \ref{fig:testing_flowchart} below illustrates our overall process, which is divided in two Phases: 
\begin{itemize}
    \item A Training Phase, in which we train PassGAN on a given dataset and sample password candidates from the trained model.
    \item A Testing Phase, in which we use HashCat to try and crack a subset of the passwords in the libero leak.
\end{itemize}

\begin{figure}[H]
\centering
    \includegraphics[scale=0.8]{figures/testing_flowchart_fixed.png}
    \caption{A diagram illustrating our general workflow for evaluating PassGAN, including Training and Testing}
    \label{fig:testing_flowchart}
\end{figure}    

\subsection{Training the PassGAN system}
In order to train PassGAN we took the plain text passwords from the libero leak (667,714 passwords) and split them into two groups: 80\% of the passwords went into our Training Set, while the remaining 20\% of passwords went in the Testing set: The Training set is used to train PassGAN as can be seen in Step 1 of figure \ref{fig:testing_flowchart}, while the passwords in the Testing set are hashed with MD5 to serve as a cracking target for HashCat (this process can also be observed in the code snippet at the end of the previous section).

PassGAN can additionaly be trained on Natural Language data, shown in a dotted box in figure \ref{fig:testing_flowchart}; elements shown within a dottet box are considered optional, they can be included or not. Natural Language data will be further discussed in secion \ref{subsec:nl-testing}.

Once PassGAN is done training with a given dataset, it produces a training model that can be used to generate candidate passwords via sampling (Step 1.1 in the figure). This sample of candidate password will be the PassGAN wordlist used with Hashcat when testing, and we can choose to generate more or less candidate passwords when sampling: the tests listed in table \ref{tab:test-set} use a wordlist of 1 million password candidates, while tables \ref{tab:passgan-big} and \ref{tab:nl-results} use a wordlist of roughtly 14 million password candidates.



%In order to train the PassGAN system, we extracted the plain text passwords from the original file (which was formatted in JSON as it contained personal information beyond just the emails and passwords of the users), and split it up in two sets: a training set containing 80\% of the passwords (534,171 entries) and a testing set containing 20\% of the passwords (133,542 entries).
%We used the model trained on the bigger set to generate 1 million samples from PassGAN, that then served as the basis from our initial testing.

Below is a small sample of password candiddates generated by PassGAN, taken from the 1 million wordlist: %Maybe i should take a sample from the last checkpoint instead?
\begin{verbatim}
22052906    mariamuna   �iaup63     poldetta53
001178      04051987    topolapa    aenara
rotsanera   princone    fadyreda21  giugki
dimonepa72  12orlomon   deb57828    belnabola
marcanalla  ACADCNA�N   leegenniki  vicaletto
\end{verbatim}

%Give a small eexample of one or two particular words and words in italian that resemble them.
What we found particularly interesting we that while the samples showed the typical patterns exhibited by user-generated Passwords (numbers appended or pre-pended to words, so-called leet speak etc..), a majority of them also seemed to mimic Italian words and phrases: most of these words were meaningless, but nonetheless we speculated that the GAN might be trying to \enquote{learn} Italian as a side-effect of generating samples close to the input data distribution.
As an example the string \texttt{vicaletto} resemble the italian word \enquote{vicoletto} meaning \enquote{a narrow street}, while the string \texttt{princone} sounds vaguely like the italian word for pickaxe, \enquote{piccone}. 

\subsection{Testing the PassGAN system}
The testing process is depicted in Step 2 of figure \ref{fig:testing_flowchart}: In it we try to crack a subset of the Libero leak passwords with hashcat, using a wordlist and optionally a set of mangling rules.
During our testing we tried to follow the same sort of methodology that Hitaj et al.\cite{PassGAN} used in their paper: as such, our cracking target is a subset of the Libero Testing set composed of passwords with a lenght of 10 characters or less: This was to better use PassGAN, as during training we set the maximum sequence lenght to 10.
We believe this sub-set of ten-character is still representative of the whole set, as the vast majority of passwords in the Testing are 10 characters or less: specifically 111,994 or 83.86\% of passwords in the Testing set have this characterisitc. We will call this subset of the testing set \emph{test\_10c}, and it will act as the craking target in all of our tests. 

Touching upon thee wordlist and rules used, they are also depicted in figure \ref{fig:testing_flowchart}: for forrmer we use either the \emph{RockYou} wordlist or a wordlist generated by PassGAN, while for the latter we use either the \emph{best64} or \emph{generated2} rule sets. Both the RockYou wordlist and the two rulesets were chosen because they were used inHitaj et al. \cite{PassGAN}, but also because they are widely used, and understood to be among the more efficient wordlists available to an attacker. We believe that they demonstrate rather well what rule-based password crackers can do without further language specific optimizations such as those that PassGAN provides.

The RockYou wordlist contains more than 14 million strings, all of which come from various data breaches that have been compiled into one resource; it contains password of varying length and complexity, and as such should provide a good performance baseline for rule-based crackers. The best64 rules file contains around 70 mangling rules, while the genrated2 rules file contains more than 65,000. 

The tables below will illustrate the results of our tests cracking the test\_10c set with different combination of Wordlists and rules. To clarify, we decided early on to use mangling rules also when testing password samples from PasGAN, since they greatly increased the number of passwords found. %If you wanna put the results for huge passgan wordlist here is where they go.

As a final note, Hashcat's default behaviour is to cache passwords found during an attack in what the Hashcat manual calls a \emph{potfile}, so that if an attacker chooses to approach the target using a different wordlist or a different set of rules, no time is wasted cracking passwords that were already found previously; because using different rule sets can greatly affect the efficiency of cracking, this form of caching allow an attacker to crack more passwords by combining different approaches.
In order to test the performance of each combination separately, we disabled caching of cracked passwords between cracking attempts: this was done with the HashCat command showed in the code snippet in section \ref{sec:libero}, and it allowed us to evaluate the performance of different combination of wordlists and rules singularly, but it also translates in a lower overall number of passwords found; tables \ref{tab:test-set}, and \ref{tab:passgan-big} show show tests done with caching disabled, while the figures obtainable with caching will be mentioned at the very end of this section. 

\begin{table}[H]
\begin{tabular}{|l|c|c|c|}
\hline
\textbf{\emph{Wordlist/Ruleset}} & \textbf{-} & \textbf{Best64} & \textbf{Gen2} \\ \hline
\textbf{Rockyou}          & 19,215 (23.78\%) & 30,170 (37.33\%) & 59,134 (73.18\%) \\ \hline
\textbf{PassGAN}          & 4,637 (5.74\%) & 11,053 (13.68\%) & 34,896 (43.18\%) \\ \hline
\end{tabular}}
\caption{A comparison of RockYou and PassGAN's cracking performance}
\label{tab:test-set}
\wfill
\hfill

% \begin{tabular}{|l|c|c|c|}
% \hline
% \textbf{\emph{Wordlist/Ruleset}} & \textbf{-} & \textbf{Best64} & \textbf{Gen2} \\ \hline
% \textbf{Rockyou}          & 70,550 (25.36\%) & 115,339 (41,47\%) & 202,624 (72.85\%) \\ \hline
% \textbf{PassGAN}          & 15,924 (5.72\%) & 40,659 (14.62\%) &  134,114 (48.22\%) \\ \hline
% \end{tabular}}
% \caption{Number of passwords found in the Training set}
% \label{tab:train-set}
\end{table}

Table \ref{tab:test-set} shows that PassGAN performs rather poorly in comparison: we attributed this shortcoming to two main factors: firstly the size of the wordlist generated by PassGAN, which is 1 million entries as opposed to the 14+ million entries of RockYou, the second factor is the quality of the PassGAN wordlist; both of these factors play into each other, because while the rules provide great advantages in terms of password found, ultimately they just modify the entries in the wordlist. Thus we might say that the whole system is limited by the capabilities of the wordlist that is used.

In order to test this hypothesis we used our existing PassGAN model to generate a bigger wordlist with as many entries as as RockYou, and performed a second test on the Testing password set:
\begin{table}[H]
\centering    
\begin{tabular}{|l|c|c|c|}
\hline
\textbf{\emph{Wordlist/Ruleset}} & \textbf{-} & \textbf{Best64} & \textbf{Gen2} \\ \hline %First row is wrong, you copied from Training. 
\textbf{Rockyou}          & 19,215 (23.78\%) & 30,170 (37.33\%) & 59,134 (73.18\%) \\ \hline
\textbf{PassGAN (14 million entries)}          &  10,735 (13.28\%) & 20,638 (25.54\%) & 48,751 (60.33\%) \\ \hline
\end{tabular}}
\caption{PassGAN cracking performance when using a bigger wordlist} 
\label{tab:passgan-big}
\end{table}

As table \ref{tab:passgan-big} shows, PassGAN found roughly 15\% more passwords when using a bigger wordlist: this leads us to believe that wordlist size and quality does constitute a performance bottle-neck. It should be also noted that when compared with the original, 1 million string wordlist, we found a significant 40\% overlap in the strings that were generated. Addressing the problem of quality of the wordlist, we speculate that one of the reasons for the gap in efficiency between PassGAN and RockYou might be the fact that the strings generated by PassGAN don't follow grammatical rules, and this might hinder the efficacy of the base wordlist. Mangling rules help greatly in finding passwords, but ultimately they are patterns applied to the base wordlist entries: if the base wordlist is less efective becasue it does not follow grammatical rules, the combined efficiency will go down.
This hypothesis will be proven wrong in section \label{subsec:nl-testing}.

When running rockyou and PassGAN sequentially on the testing set using the gen2 ruleset, we get the overall best results with 79.01\% of password matched (63,847 passwords): PassGAN found an additional 4,713 passwords that were not matched by RockYou, consisting of mostly Proper names and italian words.

\subsection{Natural language corpora} \label{subsec:nl-testing}
%Find a more reader-friendly explanation for grammatical correctness, especially at the beginning  when defining the goal.
Our next step was to include natural language data in the input and re-train the model, in the hope that this new input data would help the system generate more grammatically correct strings and thus improve the number of passwords matched.

%Talk about how the words in the corpora are sorted and how you scrambled them.%Talk about how the words in the corpora are sorted and how you scrambled them.%Talk about how the words in the corpora are sorted and how you scrambled them.%Talk about how the words in the corpora are sorted and how you scrambled them.%Talk about how the words in the corpora are sorted and how you scrambled them.%Talk about how the words in the corpora are sorted and how you scrambled them.%Talk about how the words in the corpora are sorted and how you scrambled them.
For this purpose we have chosen two different corpora of Italian Natural Language samples: 
\begin{itemize}
    \item The Repubblica corpus: A corpus of words extracted from the italian newspaper "Repubblica", taken from articles published between 1985 and 2000 (roughly 380 million words).\cite{repubblica_corpus}
    \item The ItWaC corpus: A large 2 billion word corpus obtained by crawling internet sites under the \texttt{.it} domain. \cite{itwac_corpus}.  
\end{itemize}
%You can increase the number of guesses by generating more samples and increasing the size of the wordlist, though there might be diminishing returns at some point.
%The nice thing about PassGAN is that you have theoretically infinite wordlists, even if the quality might be up and down.
Both corpora were generated using the NoSketch Engine online tool \cite{nosketch_engine}, and we decided to sample two dataset from each corpus with the same number of entries as the libero password set. This was done to ease the process of organizing data for training PassGAN, and also because we were concerned about the impact that different ratios of password data to natural language might have on the resulting model.
Previous research \cite{Melicher2016} has suggested that introducing Natural Language data into a system trained on passwords tends to generate a lot of noise during training, and our results turned out to be in line with that paper.

The data preparation process for the Natural Language datasets was very similar to what we did for the libero set, we divided our datasets into 80\%-20\% subsets: while this was not strictly necessary since we did not intend to crack the natural language datasets (that was why we originally held back 20\% or our password data), we went ahead anyway in case we needed to do that later. 

In order to test the impact of a large quantity of natural language data, we compiled two wordlists for each corpus for a total of four: for each corpus we had one shorter wordlist and one longer one.
The shorter wordlists contained all of the password data we used in our first model plus 50\% of the respective NL corpus, while the longer ones contained all of the password data and all of the NL data for their corpus.
The results of our tests on all four wordlists are shown in table \ref{tab:nl-results} below. It should be also noted that in order to maximize the number of passwords found, all tests where run using the Gen2 rule set and using the same number of PassGAN samples as table \ref{tab:passgan-big} for this reason the numbers shown in the following table should be mostly compared with table \ref{tab:passgan-big}.

Our hypothesis was that we might see a decrease in model performance in the second case, since the full NL datasets contains as much data as the password one and this might lead to noise, resulting in a model with less focus on generating passowrds as opposed to generating NL samples. While that is technically true, overall table \ref{tab:nl-results} shows that the introduction of NL data does not seem to have any positive impact of cracking performance.

% Table \ref{tab:nl-results} below shows the results of our tests for both datasets: the entries labelled with 50\% are composed of the 80\% of the libero set and 50\% of the natural language data from that source, while the other two entries are the results of PassGAN when trained on 80\% of the libero set and all of the natural language data instead. All tests were persormed using the generated2 ruleset.\\
% After re-training PassGAN using natural language samples with the process outlines above, we find that the inclusion of this new NL data does not seem to have much of an effect of the cracking performance of PassGAN;
\begin{table}[H]
\centering
\begin{tabular}{|l|l|}
\hline
\textbf{Repubblica 50\%} & 51,232 (63.40\%) \\ \hline
\textbf{Repubblica 100\%} & 48,651 (60.20\%)  \\ \hline
\textbf{ItwaC 50\%} & 50,133 (62.04\%)  \\ \hline
\textbf{ItWaC 100\%} & 47,646 (58.96\%)  \\ \hline
\end{tabular}
\caption{Number of passwords found in the Testing set using PassGAN trained on passwords and Natural Language data}
\label{tab:nl-results}
\end{table}

As we can see the results of training with natural language data are very similar to just using PassGAN+gen2 as shown in table \ref{tab:passgan-big}; there is a slight improvement of 2-3\% when using the datasets containing 50\% of language data, but we do not believe it is a significant change. This result seems to be in line with Melicher et al.\cite{Melicher2016}, and it seems to show us that natural language data does not have much of an impact on the output of PassGAN. 
\newpage
If we take a random sample from the Repubblica 50\% set, the one that performed the best even if marginally so, we can see that there has not been a substantial improvement in the grammatical correctness of the entries:
\begin{verbatim}
c3b243dc    fetemanio   carcilla    RATS1203
valentto    itefis82    carmoinx    gip1904
sederonco   elestarsia  220689      Q�unkYYTx84
n!utelo     kedea       Colpudari   rich1770
aletsa      simoshero   gidni12     1uva1[78
\end{verbatim}    

The words that PassGAN are still meaningless and does not seem to have a higher degree of grammatical correctness, the only change we have been able to observe is in the fact that more strings seem to be composed of letters exclusively.

This does not discredit the use of Natural language as a whole, but points us toward the fact that natural language generation and password generation are two different workloads that may not be accomplished well with the same machine learning system: it seems that introducing Natural Language data into PassGAN simply adds noise to the system and does not contribute to the system's performance.
The results in table \ref{tab:nl-results} and the effectiveness of mangling rules as a cracking tool also hint at the idea that user-generated passwords may be defined by their patterns more than by their language of origin. 

\clearpage
\section{Discussion}

In this coming section we are going to discuss some of the shortcomings of this thesis and talk about avenues for future work.

While we were overall pretty happy with the results we obtained from our testing, we feel that there are some deep-seated problems in our arguments that we need to address:

First, its the fact that we did not try to crack the Libero passwords using the pre-trained PassGAN model that Hitaji et al. \cite{PassGAN} used in their paper: we think this is a rather important omission, 
as doing that would have given us a base-line performance that we could compare our PassGAN model to in order to determine what were the effects of using a model trained on largely english-language passwords to crack a dataset of non-english language passwords. We feel such an experiment would have given us much better footing when we claim to be evaluating PassGAN using italian passwords. As it stands we are 
still evaluating PassGAN in that way, but not having the reference of that performance comparison makes our evaluation less meaningful

Since the beginning the goal of this paper was to test the impact of natural language on PassGAN, and because we were so focused on that objective we skipped over the important experiment outlined above; 
speaking more broadly we gradually convinced ourselves that natural language would have a positive impact on the model's performance, and in a way we shaped the paper expecting to arrive at that conclusion: 
when we got back the results and realised that the numbers disproved our hypothesis, we were shocked and we gradually realized that our contributions were not as solid as we had hoped.

Speaking of avenues of further work, there were some aspects of our thesis that we would have liked to expand upon:
When discussing the impact of Natural Language data on PassGAN we mentioned that different ratios of password data to Natural Language data affect the model performance in different ways: while we observed that PassGAN performs marginally worse when using equal parts of password data and NL data, it would have been interesting to explore this further: we would have been interested in researching what is the ideal ratio to maximize PassGAN performance. \\
A similar argument might be had concerning wordlist size for PassGAN: In section \ref{sec:testing_and_evaluation} we briefly touched upon using even bigger PassGAN wordlists; We believe it would be natural to wonder whether there are diminishing returns concerning wordlist sizes, i.e if there is a point at which adding more words does not lead to a meaningful increase in the number of password size.
In our experience it takes a huge number of password candidates sampled from PassGAN in order to crack passwords without the use of rules, and there might be a point at which PassGAN might stop producing unique password patterns leading to a slump in the number of password found. One of the advantages of machine learning systems such as PassGAN is that they can keep producing new password candidates almost indefinitely, so we feel that such research might be beneficial in understanding the limits of these systems and also their applicability. Regardless of this potential, as we showed in section \ref{sec:testing_and_evaluation} PassGAN can be useful as a specialist tool to target complex passwords when used in conjunction with rule-based password crackers.

Looking back at our work in this paper, we also realise that it might have been a good idea to attempt to reproduce the findings in the PassGAN paper: it would have helped us to better ground our discussion in the context of their paper, but overall we do not think such attempts at reproduction are strictly necessary: while we have certainly looked at Hitaji et al.'s data to give us a broad indication of performance, we were using a different dataset and we don't believe that it would have been meaningful to directly compare our results.

  

\clearpage
\section{Conclusion}
%Small problem: conclusions 2 and 4 are not mine, im pretty sure they are things that the PAssgAN paper said before me.
In this thesis we have looked at the impact that language has on PassGAN in various ways:
To conclude our thesis, we would like to answer the questions posed in our problem formulation all the way back in section \ref{sec:introduction}.

\begin{itemize}
\item \emph{How does PassGAN perform when cracking Italian passwords?} %Implies: is it better or worse when using the pre-trained model as opposed to a new model trained on italian passwords?

Overall we can say that PassGAN performs worse than rule-based password crackers when looking at the total number of passwords found as we show is section \ref{subsec:passgan-testing}, but this fact was already true for Hitaj. et al. \cite{PassGAN} when considering PassGAN alone.
That said however there are also benefits to using PassGAN, as it seems useful in finding particularly complex passwords.

Overall we sadly don't have a full picture of PassGAN's performance with Italian passwords, because as we said in section \ref{sec:discussion} we were not able to test PassGAN's pre-trained model used in \cite{PassGAN} against the Libero passwords.

\item \emph{Does difference in language have an impact on the passwords found by PassGAN and state-of-the-art password crackers?}

We would say yes: even if PassGAN perform worse than traditional password crackers in terms of total number of passwords found, we argue that there is a difference in the kinds of passwords that can be found by using PassGAN over HashCat: this was particularly evident when running the two together as we did in section \ref{subsubsec:potfile-enable}.      

\item \emph{How does the inclusion of natural language data during training affect PassGAN's performance?}

From our experience, the inclusion of natural language data does not seem to have a positive impact on the performance of PassGAN: in fact, it had a slightly negative impact. Overall PassGAN's Performance using NL data can broadly be considered on par with a PassGAN model trained with just passwords.
 
\item \emph{Ultimately, can PassGAN be a useful tool to use when approaching password data from a particular language area?} %Or are rule-based tools always better

We believe PassGAN can indeed be a useful tool when approaching passwords from another language area: while its applicability is narrow (we would consider it more as a specialist tool, useful for approaching more complex passwords that may use specific language or refer to a specific culture), we think it can be rather useful when used in conjunction with more generalists rule-based passwords crackers. We find ourselves in agreement with Hitaj, et al. on this point.    
\end{itemize}

%Another Interesting avenue of research could be to test the limits of PassGAN as a language-sensitive specialist cracker: specifically we think it might be interesting to tackle the subject of passphrases using PassGAN: one might make the case that by their nature, passphrases are more subject to grammatical and syntactical rules, and it would be interesting to repeat the experiments in this thesis using passphrases as a dataset. \\
%As \cite{Melicher2016} states, rule-based password crackers tend to have a ceiling on the length and complexity of passwords they can efficiently crack: A similar study using passphrases might also test the limits of rule-based tools in such a way, that we might find PassGAN to be a more effective tool for this particular niche. The big problem we see with this research is the lack of data: passphrases are not usually widely employed by companies in their password policies, and there is a lower chance of leaked passphrases datasets being available to the public.

\clearpage
\thispagestyle{empty}
\printbibliography[heading=bibintoc]
\label{bib:mybiblio}
\appendix
%commented for now for faster compilation
%\input{appendix/appendixes.tex}
\end{document}