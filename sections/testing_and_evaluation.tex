\section{Testing and evaluation}
In this next section, we are going to evaluate the performance of the PassGAN system, as well as give a brief overview of our methodology and the experimental setup.

\subsection{Experimental setup}
Both training of the GAN and testing were carried out on a Linux machine running Debian stretch, equipped with a GTX 1070 graphics card, an i5-2500 CPU and 8GB of system memory. 
With this card, training of PAssGAN took roughly eleven hours.
PassGAn was run on Python 2.7, running Tensorflow 1.12.0.\\
For testing we used the latest stable version of HashCat as of this writing (version 5.1.0) and we observed a peak speed between 850 and 860MH/s (for reference, 1MH/s is equivalent to 1 million candidate passwords hashed and compared per second) %Reference  %The definition of MH is probably wrong

\subsection{Testing the PassGAN system}
In order to train the PassGAN system, we extracted the plain text passwords from the original file (which was formatted in JSON as it contained personal information beyond just the emails and passwords of the users), and split it up in two sets: a training set containing 80\% of the passowrds (534,171 entries) and a testing set containing 20\% of the passwords (133,542 entries).
We used the model trained on the bigger set to generate 1 million samples from PassGAN, that then served as the basis from our testing.

Below is an example of samples that were produced by PassGAN at the very end of training: %Maybe i should take a sample from the last checkpoint instead?
\begin{verbatim}
22052906    mariamuna   �iaup63     poldetta53
001178      04051987    topolapa    aenara
rotsanera   princone    fadyreda21  giugki
dimonepa72  12orlomon   deb57828    belnabola
marcanalla  ACADCNA�N   leegenniki  vicaletto
\end{verbatim}

%Give a small eexample of one or two particular words and words in italian that resemble them.
What we found particularly interesting was that while the samples showed the typical patterns exhibited by user-generated Passwords (numbers appended or pre-pended to words, leet speak etc..), a majority of them also seemed to mimic Italian words and phrases: most of these words were meaningless, but nonetheless we speculated that the GAN might be trying to "learn" Italian as a side-effect of generating samples close to the input data distribution.

During our testing we tried to follow the same sort of methodology that Hitaj et al.\cite{PassGAN} used in their paper: as such, in our testing we used a subset of both our training  and our testing set, composed of passwords with a length of 10 characters or less. This was done in order to focus the scope of PassGAN, as during training we set the sequence length to 10 %Bullshit you just wanna give yourself a leg up. %Add number of passwords in the 10c sets %MAYBE the oassgan ppl trained the GAN on a 10c datasets as well? maybe i should re-train to test this.

We beleive this sub-set of ten-character is still representative of the whole set, as the vast majority of passwords in the Libero set are 10 characters or less: 461,442 (86.38\%) in the Training set, and 111,994 (83.86\%) in the Testing set. 

In order to test the Performance of PassGAN, we compared it with other wordlists and rules commonly used in rule-based password crackers: Specifically we chose \emph{RockYou} as our wordlist, alongside \emph{best64} and \emph{generated2} as our rules; All of them were chosen because they were also used by Hitaj et al.\cite{PassGAN}.
%Mention how rockyou is not only a erm of comparison but also your baseline and the "standard way of approaching password cracking".
The RockYou wordlist contains more than 14 million strings, all of which come from various data breaches that have been compiled into one resource; it contains password of varying length and complexity, and as such should provide a good performance baseline for rule-based crackers. The best64 rules file contains around 70 mangling rules, while the genrated2 rules file contains more than 65,000. 
Finally in order to test the output of PassGAN we used a wordlist containing one million samples generated by the model we trained earlier.

The tables below will illustrate the results of our tests, performed on both the Training set and the Testing set using the wordlists and rules described above. To clarify, we decided early on to use mangling rules also when testing password samples from PasGAn, since they greatly increased the number of passwords found. In order to test the performance of each combination separately, we disabled caching of cracked passwords between cracking attempts: this allowed us to evaluate the performance of different combination of wordlists and rules singularly, but it also translates in a lower number of passwords found; if we combined these techniques we might crack an greater percentage of the total passwords.

\begin{table}[H]
\begin{tabular}{|l|c|c|c|}
\hline
\textbf{\emph{Wordlist/Ruleset}} & \textbf{-} & \textbf{Best64} & \textbf{Gen2} \\ \hline
\textbf{Rockyou}          & 19,215 (23.78\%) & 30,170 (37.33\%) & 59,134 (73.18\%) \\ \hline
\textbf{PassGAN}          & 4,637 (5.74\%) & 11,053 (13.68\%) & 34,896 (43.18\%) \\ \hline
\end{tabular}}
\caption{Number of passwords found in the Testing set}
\label{tab:test-set}
\wfill
\hfill

\begin{tabular}{|l|c|c|c|}
\hline
\textbf{\emph{Wordlist/Ruleset}} & \textbf{-} & \textbf{Best64} & \textbf{Gen2} \\ \hline
\textbf{Rockyou}          & 70,550 (25.36\%) & 115,339 (41,47\%) & 202,624 (72.85\%) \\ \hline
\textbf{PassGAN}          & 15,924 (5.72\%) & 40,659 (14.62\%) &  134,114 (48.22\%) \\ \hline
\end{tabular}}
\caption{Number of passwords found in the Training set}
\label{tab:train-set}
\end{table}

As table \ref{tab:test-set} and \ref{tab:train-set} show, the relative amount of password found with each technique is roughly consistent between the two sets.
One point is immediately evident, and that is that PassGAN performs rather poorly in comparison: we attribute this shortcoming to two main factors: firstly the size of the wordlist generated by PassGAN, which is 1 million entries as opposed to the 14+ million entries of RockYou, the second factor is the quality of the PassGAN wordlist; both of these factors play into each other, because while the rules provide great advantages in terms of password found, ultimately they just modify the entries in the wordlist. Thus we might say that the whole system is limited by the capabilities of the wordlist that is used.

In order to test this hypothesis we used our existing PassGAN model to generate a bigger wordlist with as many entries as as RockYou, and performed a second test on the Testing password set:
\begin{table}[H]
\begin{tabular}{|l|c|c|c|}
\hline
\textbf{\emph{Wordlist/Ruleset}} & \textbf{-} & \textbf{Best64} & \textbf{Gen2} \\ \hline %First row is wrong, you copied from Training. 
\textbf{Rockyou}          & 70,550 (25.36\%) & 115,339 (41,47\%) & 202,624 (72.85\%) \\ \hline
\textbf{PassGAN (14 million entries)}          &  10,735 (13.28\%) & 20,638 (25.54\%) & 48,751 (60.33\%) \\ \hline
\end{tabular}}
\caption{Number of passwords found in the Training set with a bigger PassGAN wordlist} 
\label{tab:passgan-big}
\end{table}

As table \ref{tab:passgan-big} shows, PassGAN found roughly 15\% more passwords when using a bigger wordlist: this leads us to believe that wordlist size and quality does constitute a performance bottle-neck. It should be also noted that when compared with the original, 1 million string wordlist, we found a significant 40\% overlap in the strings that were generated. Addressing the problem of quality of the wordlist, we speculate that one of the reasons for the gap in efficiency between PassGAN and RockYou might be the fact that the strings generated by PassGAN don't follow grammatical rules, and this might hinder the efficacy of the base wordlist. Mangling rules help greatly in finding passwords, but ultimately they are patterns applied to the base wordlist entries: if the base wordlist is less efective becasue it does not follow grammatical rules, the combined efficiency will go down.

\subsection{Natural language corpora}
Our next step was to include natural language data in the input and re-train the model, in the hope that this new input data would help the system generate more grammatically correct strings and thus improve the number of passwords matched.

For this pourpose we have chosen two differnt coorpora of Italian Natural Language samples: 
\begin{itemize}
    \item The Repubblica corpus: A corpus of words from the italian newspaper "Repubblica", taken from artciles published betweeen 1985 and 2000 (roughtly 380 milllion words).\cite{repubblica_corpus}
    \item The ItWaC corpus is a large 2 billion word corpus obtained by crwaling internet sites under the \texttt{.it} domain. \cite{itwac_corpus}.  
\end{itemize}



