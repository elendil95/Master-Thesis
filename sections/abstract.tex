\thispagestyle{empty}
\begin{abstract}
The goal of this thesis is to explore what impact language has on password cracking (if any) when using Generative Adversarial Networks (GANs) to crack passwords.
In it we build on existing research by taking an existing GAN-based password cracker, and testing it with a dataset of leaked user passwords from Italy. 
By comparing the performance of the resulting model with widely-used wordlists based primarily on English-language sources, as well as state-of-the-art rule-based tools, we hope to get insights into the impact that different grammatical structures have on both the performance of the GAN model, and by extension on its performance as a password cracking tool.

To that end we also wish to explore whether including Italian language corpora during training is decremental to the network's performance (as some prior research seems to indicate), or if it can prove beneficial in this particular context. 
\end{abstract}
