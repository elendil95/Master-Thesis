%TODO: Add an introduction to the concept of rules where you see fit and talk more about keyspace and set theory (ugh)
%Also the bulk of changes on the AI section will probably be talking about PassGAN specifically and not just the idea of GANs
%oh and FIX YOUR REFERENCES FOR FUCK'S SAKE
\section{Related Work}\label{sec:related_work}
Both Password Cracking and deep learning are active areas of research developing at a rapid pace.

In this section, we aim to give an overview of the relevant knowledge in these areas as it relates to our thesis.

\subsection{Password Cracking}
Password cracking has been around for a long time, and while technology has evolved greatly and continues to do so, the basic concepts have remained relatively similar:
Back in 1979, Morris and Thompson were already aware of different ways to attack passwords and defend against such attacks \cite{Thompson1979}.

At the base of all password attacks is the concept of \emph{key space}, i.e the set of all possible passwords of a certain length that use a specific character set \cite{Thompson1979,hash_cat_mask_attack}. 
%Key sooace is theoretically infinite. add that.
%Maybe you could use that paper full of set thory math. Just check the scholarship beforefand.
In the case of the password \texttt{password}, the key space can be expressed $26^8$ (i.e The number of possible characters to the length of the string).

Key space is important because it is the basis from which password cracking techniques work off: if the password is sufficiently complex, an attacker will have to do an exhaustive search of the key space in order to find the password. 
It follows then, that one of principles of password strength is to make the search space big enough to be impractical to search through with current hardware.
This is the reason why users are commonly advised to use a variety of different characters classes in their passwords.

To ground our discussion of search space in the context of cracking, we should address how passwords are stored in the first place: As Morris and Thompson explain in their paper, the simplest approach is to store the users' passwords in a file or database as they are entered: this is a bad idea as any software bug that causes an accidental disclosure will leave the users exposed, and also because any privileged user can simply look up other users' passwords.

A better approach would be to encrypt the user's password, and store the cypher text: when a user logs in, the string they typed is compared with the cypher text and access is granted if it matches. This encryption is commonly achieved through a key derivation function or one-way cryptographic hash function, and Morris and Thompson use the DES algorithm in their paper. %Check that the terminology is correct.
Simply put, a cryptographic hash function is a mathematical function that, given a seed or key, encrypts a string in a predictable way; another feature of such function is that converting cypher text back into plain text should be exceedingly difficult, thus making reversing the process unfeasible.
This propriety however changes as the amount of available computing power increases, and any given hashing algorithm will eventually become insecure.

When a strong hashing algorithm is used, an attacker will theoretically have to do a key space search in order to crack the users' password: this process will take even more time, as every candidate string must be first encrypted with the same algorithm and then looked up in the database.

Encryption alone does not solve the problem however, because in reality an attacker would not have to resort to brute-forcing in a majority of cases.

Because the hashing algorithm needs to always output the same result for a given input string, the attacker can start comparing the hashes in the database and draw some conclusion: if some hashed appear many times, they probably hold the plain text of some very common passwords; If the attacker cracks those first and starts looking for similar patterns, he may crack a sizable number of the passwords contained in the database.

This leads us to our next attack strategy, which is to build a table with pre-computed hashes for all the most common passwords: this technique saves a lot of time since we do not need to encrypt every candidate password before comparing it with the database, and instead we merely look up the hashes from the table in the database. 

These tables are commonly referred to as \emph{Rainbow Tables}, and they are a simplified version of the rule-based techniques covered in section \ref{hash_and_jtr}.

Rainbow tables can be defeated by using \emph{password salting}: salting works by generating a random string that is then appended to the user's password before it is encrypted (mainly when the user first creates the account). The authors in \cite{Thompson1979} use a 12-bit random number as their salt, but modern sources suggest the use of longer and more complex salts \cite{NIST_2017}; by generating a unique salt per each user (or even better, per each individual password), Rainbow tables and correlation attacks will be rendered useless as its no longer possible to generate hashes of passwords that will work on the target; one might attempt to generate tables that contain salts, but this is impractical as it would mean brute-forcing the key space of the salt.
This can be done however in cases where the salt is very short, or alternatively when the system administrators have chosen one fixed salt string with which every password is processed.
The latter mistake its particularly grave, as it defeats the purpose of having salts in the first place.
   
Its important to store the salt value in a proper manner however, because salted passwords may still vulnerable to rule-based attacks if the salt is leaked alongside the user's passwords. \newline

Let us suppose that an attacker has access to both the hashed passwords of the users and the salts that each password uses: it is conceivable that one might extract all the salts from the database, put them in a dictionary, and use them in a rule-based dictionary attack such as the ones that will be described in section \ref{hash_and_jtr}: if the salt for each user is stored in the database, we do not need to brute-force the key space of the salt: every salt belongs to one of the users, hence we merely need to try them all sequentially on every password candidate. 

This would increase the difficulty of the attack by an order of magnitude and is unlikely, but it might be possible to carry out such an attack in a reasonable time given enough computing power. A remedy for such scenario is offered in \cite{NIST_2017} and will be discussed further down. %FUCK ME WITH A RUSTY SPEAR, I HAVE NO IDEA IF THIS IS BS I JUST MADE UP!

For a more current example of how these security practices have evolved since 1979, we can look at the National Institute of Standards and Technologies' standard SP 800-63-3.
This standard is intended to provide security guideline for information systems within the US government, and was released in June 2017.

Document SP 800-63-3B \cite{NIST_2017}, provides guide-lines on how user secrets (passwords and/or PINs) should be stored.\\
They suggest that user password should be between 8 and 64 characters, but advise against enforcing password policies concerning the composition of passwords; Instead, they suggest that users passwords be checked against a list of leaked passwords and dictionary words before they are accepted, in order to avoid the use of weak passwords.

When it comes to encryption, they suggest the use of the PBKDF2 and Balloon algorithms; Other hashing algorithms listed as suitable for password encryption include HMAC and SHA-3.

Furthermore, they recommend that all passwords be salted with at least a 32bit quantity; to our understanding the standard calls for a unique salt for each user, as opposed to a unique salt for each password. The salts should be stored alongside the hashed password of the user.

In order to foil dictionary attacks in cases where the attacker has access to the users salts alongside the hashes, the system should perform an additional iteration of the encryption algorithm using a different 112bits salt that is secret, and should be stored on separate hardware form the main data.\newline 
If we were to apply such process before encrypting the users password with the standard 32bit salt, this would effectively neutralize all dictionary attacks even when the attacker factors the salts into the process.

To conclude this discussion of password storage and security, we will briefly touch upon our main dataset used in the thesis: the Libero dataset.

The Libero dataset is a JSON formatted document that contains information on roughly 700 thousand users of the italian email provider Libero Mail \cite{libero_leak}.\\
Each user record contains various fields such as  user ID, user name, the associated email address and the user's password in plain text.
These plain text passwords served as the basis for all our experiments (detailed in section \ref{sec:testing_and_evaluation}), in which we attempted to crack the Libero dataset with rule-based password crackers and PassGAN. Because the passwords were provided in plain text, we had to hash them with MD5 in order to crack them.
The fact that our dataset did not contain hashed passwords presented a problem, and called into question the authenticity of the data in our possession: we will go into further detail with this  and the choices we made regarding the cracking process in section \ref{sec:libero}.  

Now that we have a better notion of how passwords are stored, the next section will address how passwords are commonly cracked: specifically, we will use state of the art rule-based password crackers in order to explore the cracking process; to briefly elucidate on how these tools fit in the greater scope of the paper, PassGAN and other such Deep Learning systems can be seen as a complement to rule-based password crackers, as their output is used as input for rule-based tools in order to practically crack passwords. The interplay between these two sides will be explored in section \ref{sec:testing_and_evaluation}.

\subsubsection{Rule-Based password crackers} \label{hash_and_jtr}
%Define rules firrst in the introduction

Two common tools for password cracking are John The Ripper and \break \mbox{HashCat}\cite{john,hash_cat}: these tools take advantage of common patterns in user behaviour in order to optimize the cracking process: Both tools have a variety of techniques an attacker can employ, and we will briefly exemplify their capabilities using HashCat as an example \cite{hash_cat_wiki}:

In both tools there are three categories of attacks that can be carried out: brute-force attacks, dictionaries attacks and rule based attacks: these can be combined and tweaked in various ways depending on the desired result, and there is a degree of overlap between each

\begin{itemize}
\item Brute-force attacks try ever possible combination of characters within a defined pattern, and thus closely relate to keyspace search. 
\item Dictionary attacks use a dictionary of words or passwords as a source for password candidates, that are hashed and checked against the target to see if any of them match
\item Rule-based attacks are an evolution of Dictionary attacks, in which the dictionary's effectiveness is boosted by some mangling rules that change the words in ways that reflect common patterns in user-generated passwords.
\end{itemize}

HashCat distinguishes between Brute-force attacks and Wordlist based attacks, but there is no distinction between wordlist-based attacks and Rule-based attacks: instead rules are an optional parameter one can use with use with wordlist attacks to greatly increase their effectiveness.
Finally, HashCat has modes of operation which roughly reflect our three categorizations 

\paragraph{Brute-force attacks}
The simplest of these modes is \emph{Mask attack} mode, used to carry out brute-force attacks: this is modified version of key space search, meant to attack simpler passwords while shrinking the key space search. 
Instead of searching the total space for a password of length $x$, we define a simple regular pattern expressing the what character classes are there and at which position, saving us a substantial amount of processing time.
For example is we have a password like \texttt{Benjamin86} or \texttt{Iloveyou02}, we can define our pattern to be a ten-character string with eight lowercase or uppercase letters and two numbers at the end; This is referred to as a \emph{Mask} in HashCat's documentation:

In a classic brute-force attack we would deal with a search space of $62^{10}$ (or roughly $8 \times 10^{17}$ combinations), but thanks to the above-mentioned mask we can reduce our search space to around $4 \times 10^{13}$ possibilities if we assume that the first character in the string is the only one that can be uppercase.
Furthermore we can use the \texttt{--increment } option in HashCat to apply this pattern to all strings up to ten characters, allowing us to match shorter passwords that follow the same pattern.

This method is rather simplistic and not very flexible, but exemplifies some of the ways in which attackers can optimize key space attacks.

\paragraph{Dictionary attacks}
The main mode of operation oh HashCat \emph{straight} mode, that performs a dictionary attack: In this mode, the program is fed a wordlist/dictionary, and tries each entry the wordlist as a password candidate. Because of its simplicity, such a dictionary attack works best with a wordlist composed of leaked passwords; the aim is to target very common passwords and users that re-use passwords, but the effectiveness of such an attack can increase significantly depending on what wordlist is used.

Dictionary attacks can be further enhanced by combining them in various ways: one approach would be to use two wordlists and append/prepend each entry in the second one to each entry in the first; 
the second wordlist might be a natural language dictionary or a wordlist of plain-text leaked passwords. This is called \emph{Combinator attack} mode in HashCat. 

We might also want to use the output of a brute-force attempt as out second wordlist: If we use the patterns described in the Mask Attack mode to generate strings we combine with a wordlist, we will obtain a more targeted and effective version of the Mask Attack method. 
For example if we know that a good deal of the passwords we want to extract are strings with numbers appended to the end, we might run a mask that generates combinations of 0 to 4 digits and then combine the output of that with our wordlist. This is called \emph{Hybrid attack} mode in HashCat.

\paragraph{Rule-based attacks}
Finally we come to rule-based attacks: In short they are an extension of all the methods described above: as we said previously rule-based attacks do not have a dedicated mode, but instead they are often used in \emph{straight} mode alongside a wordlist in order to boost the number of passwords found by that wordlist.
Rules are flexible and allow for more thorough definition of the patterns that may appear in a password, going beyond the capabilities of a regular language used with Masks.
Patterns can be created independently of the size and characteristics of the passwords, and they are not limited to a fixed patterns; there are also flow control statements and options to apply rules only in certain conditions.
There are also options to save password candidates to memory enabling more advanced processing: saved strings can be appended to each password candidate matching certain criteria, reversed and so on...
These more advanced options allow an attacker to emulate the operation of Combinator and Hybrid attacks. 

Rules are applied to each entry in a wordlist in a similar way to a combinator attack, and multiple rules can also be be applied sequentially to the same entry in the wordlist: each rule is applied once for every entry in the wordlist, and the resulting password candidates are tried in sequence.

Rules provide a more efficient way to tackle password cracking since their greater flexibility means that an attacker need not know as much about their target. 

In order to get a better idea of how rules are used we can look at the \emph{best64} rules, which contain around 70 commonly used mangling rules:
\begin{verbatim}
## nothing, reverse, case... base stuff
:
r
u
T0
\end{verbatim}

The rules above are rather simple, they simply reverse a string, toggle each character's case or toggle the case of the first character: as we can see these rules already reflect some common patters used by users, such as using common words spelled backwards or toggling a word's case as a way to comply with password policies.
Further down in the same file we can find some more complex rules:
\begin{verbatim}
## leetify
so0
si1
se3
\end{verbatim}

The substitution rules above are design to defeat a common practice in user-generated passwords, leet speak: this is the practice of changing letters in passwords with numbers that look similar in an attempt to fit within character class requirements in passwords. 

\paragraph{Using HashCat}
In this section we will give an example of how rule-based password crackers are used in practice, using our experience with HashCat and the Libero dataset as reference. To see how we used HashCat and the Libero passwords in our thesis refer to section \ref{sec:testing_and_evaluation}.

When running HashCat (and other rule-based password cracker), an attacker needs three main pieces of data: a target file containing hashed passwords, a wordlist, and optionally a set of rules.
Additionally, one needs to know the type of hash passwords are encoded with.

Let us give an example of a HashCat command we used in our thesis and walk through the various parameters (Note that the red arrows are simply a visual aid to indicate line wrapping, to make the command fit on the page):

\begin{lstlisting}[breaklines=true,postbreak=\mbox{\textcolor{red}{$\hookrightarrow$}\space}]
hashcat -a 0 -m 0 test_10c.hash /usr/share/hashcat/wordlist/rockyou.txt -r /usr/share/hashcat/rules/best64.rule
\end{lstlisting}

Starting from the left:
\begin{itemize}
\item The first parameter \texttt{-a} indicates the attack mode, the value 0 equals to straight mode in this case for a dictionary attack.

\item The second parameter \texttt{-m} indicates the hash type, hash type number 0 is MD5. As we mentioned previously we chose to hash the plain text passwords in the Libero set to MD5 for the purposes of testing.

\item The third parameter is the target, the set of passwords to be cracked: here we used \texttt{test\_10c.hash}, which is a sub-set of the Libero dataset (further details in section \ref{subsec:passgan-testing}).

\item The fourth parameter is the wordlist to use: here we used \texttit{rockyou}, a popular wordlist composed of more than 14 million passwords from various data breaches. It comes pre-packaged with most HashCat installations.

\item The fifth parameter \texttt{-r} tells HashCat that we want to use rules.

\item The sixth parameter is the ruleset that we want to use. In this particular instance we have used \texttt{best64.rule}, the same ruleset we have taken the example rules from.    
\end{itemize}

Running this command will print the cracked passwords to standard output as they are found, in the format hash:password, alternatively its also possible to save the results to a file with the \texttt{-o} option.

We should also note that in this example we did not show how to deal with salted passwords: as we explain in section \ref{sec:libero}, the Libero dataset did not come with any salts, so we do not deal with them in this paper.
