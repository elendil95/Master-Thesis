\section{Related Work}
Both Password Cracking and deep learning are active areas of research developing at a rapid pace.

In this section, we aim to give an overview of the relevant knowledge in these areas as it relates to our thesis.

\subsection{Password Cracking}
Password cracking has been around for a long time, and while technology has evolved greatly and continues to do so, the basic concepts have remained relatively similar:
Back in 1979, Morris and Thompson were already exploring different ways to attack passwords and defend against such attacks \cite{Thompson1979}.

At the base of all password attacks is the concept of \emph{key space}, i.e the set of all possible passwords of a certain length that use a specific character set \cite{Thompson1979,hash_cat_mask_attack}. 

In the case of the password \texttt{password}, the key space can be expressed $26^8$ (i.e The number of possible characters to the length of the string).

Key space is important because it is the basis from which password cracking techniques work off: if the password is sufficiently complex, an attacker will have to do an exhaustive search of the search space in order to find the password. 
It follows then, that one of principles of password strength is to make the search space big enough to be impractical to search through with current hardware.
This is the reason why users are commonly advised to use a variety of different characters classes in their passwords.

An attacker's goal is to try to shrink the subset of the search space that needs to be examined in order to speed up the process, many times exploiting user habits or inherit characteristics of the password.\newline

Password cracking can be accomplished in many ways, and the approach often depends on the situation: the attacker might be trying to gain access to a system by cracking the password of a high-privilege user like a System Administrator, or he might be cracking a leaked password database in order to later sell the personal data contained within.

In the first scenario, we might try to learn as much as we can about the target and try to use social engineering techniques in order to obtain the password; Phishing and various kind of fraud are commonly used in such cases.

In the second scenario, we are trying to extract as many password as possible from the leaked database, and we might take advantage of user behavior in order to do so: Users tend to choose common passwords and to re-use the same password in multiple instances. If we were to attack common passwords first, we can significantly reduce the combine search space.
%Add more about passwords proper, for example about hashing algorithms and salting?

\subsection{Rule-Based password crackers}

Two common tools for password cracking are John The Ripper and \break \mbox{HashCat}\cite{john,hash_cat}: these tools take advantage of common patterns in user behaviour in order to optimize the cracking process: Both tools have a variety of techniques an attacker can employ, and we will briefly exemplify their capabilities using HashCat as an example \cite{hash_cat_wiki}: \newline

In both tools there are three categories of attacks that can be carried out: brute-force attacks, dictionaries attacks and rule based attacks: these can be combined and tweaked in various ways depending on the desired result, and there is a degree of overlap between each.\newline 

HashCat's \enquote{modes} reflect these categories, and the simplest of these is \emph{Mask attack} mode: this is modified version of key space search, meant to attack simpler passwords while shrinking the key space search: 
instead of searching the total space for a password of length $x$, we define a simple regular pattern expressing the what character classes are there and at which position, saving us a substantial amount of processing time.
For example is we have a password like \texttt{Benjamin86} or \texttt{Iloveyou02}, we can define our pattern to be a ten-character string with eight lowercase letters and two numbers at the end; This is referred to as a \emph{Mask} in HashCat's documentation:

In a classic brute-force attack we would deal with a search space of $62^{10}$ (or roughly $8 \times 10^{17}$ combinations), but thanks to the above-mentioned mask we can reduce our search space to around $4 \times 10^{13}$ possibilities if we assume that the first character in the string is the only one that can be uppercase.
Furthermore we can use the \texttt{--increment } option in HashCat to apply this pattern to all strings up to ten characters, allowing us to match shorter passwords that follow the same pattern.

This method is rather simplistic and not very flexible, but exemplifies some of the ways in which attackers can optimize key space attacks.\newline

The main mode of operation oh HashCat \emph{straight} mode, that performs a dictionary attack: In this mode, the program is fed a wordlist/dictionary, and tries each entry the wordlist as a password candidate. Because of its simplicity, such a dictionary attack works best with a wordlist composed of leaked passwords; the aim is to target very common passwords and users that re-use passwords, but the effectiveness of such an attack can increase significantly depending on what wordlist is used.

Dictionary attacks can be further enhanced by combining them in various ways: one approach would be to use two wordlists and append/prepend each entry in the second one to each entry in the first; 
the second wordlist might be a natural language dictionary or a wordlist of leaked passwords. This is called \emph{Combinator attack} mode in HashCat. 
We might also want to use the output of a brute-force attempt as out second wordlist: If we use the patterns described in the Mask Attack mode to generate strings we combine with a wordlist, we will obtained a more targeted and effective version of the Mask Attack method. 
For example is we know that a good deal of the passwords we want to extract are strings with numbers appended to the end, we might run a mask that generates combinations of 0 to 4 digits and then combine the output of that with our wordlist. This is called \emph{Hybrid attack } mode in HashCat.\newline

Finally we come to rule-based attacks: In short they are an extension of all the methods described above, and all the aforementioned techniques can also be performed using rules; however, rules are more flexible and allow for more thorough definition of the patterns that may appear in a password, going beyond the capabilities of a regular language.
Patterns can be created independently of the size and characteristics of the passwords, and they are not limited to a fixed patterns; there are also flow control statements and options to apply rules only in certain conditions.
There are also options to save password candidates to memory enabling more advanced processing: saved strings can be appended to each pasword candidate matching certain criteria, reversed and so on...

Rules are applied to each entry in a wordlist in a similar way to a combinator attack, and they provide a more efficient way to tackle password cracking since their greater flexibility means that an attacker need not know as much about their target. 
 
%They are fed a list of strings to search in the database, and a set of mangling rules: they first compare the list with the database, and then go through the process again but applying each rule to each entry in the word-list; In order to crack passwords more efficiently, words-lists usually contain lists of common passwords or real passwords from other data leaks, but they may also contain dictionary words and natural language fragments.

%The rules are given in a regular language and express common things that users do when choosing passwords (such as adding numbers at the end of a string, toggling the case of the first letter or perform common pattern substitutions like \enquote{leet speak}). because of these mangling rules, a password like \texttt{P4ssw0rd1} may seem strong, but it will be cracked quite effortlessly.
