\section{Introduction}
\subsection{Motivation}
The aim of this thesis is to evaluate the paper \enquote{PassGAN: A Deep Learning Approach for Password Guessing}\cite{PassGAN}, by testing the Deep Learning system described in the paper with a different dataset of leaked passwords from Italy\cite{libero_leak} (henceforth referred to as the Libero dataset), to test whether there are any differences in performance when PassGAN is used to crack a database from a non-english source.\newline

This dataset is composed of real user passwords belonging to the Italian email provider Libero.it, that were leaked in 2016. We believe the use of these passwords to be ethical because this particular leak has been public for a number of years. %And the company has since taken action to remediate?
\newline

The underlying tough behind this it to test whether differences in grammar and language have any noticeable effect when training the system: perhaps GANs are expressive enough that they 
can account for the different provenance of the data without any significant change, or maybe some adjustments should be made such as the inclusion of Natural Language corpora on the training data. One other possible change might be to mix password datasets from English and non-English sources.

The inclusion Natural language corpora might present its own challenges, such as the fact that some prior studies with GANs and RNNs indicates that the inclusion of such data ends up creating a lot of noise rather than improving performance\cite{PassGAN, Melicher2016}. %Check is PassGAN paper has such a claim.

These studies instead seem to suggest that one should "focus the network's effort" so to speak, by using data that is as close as possible to the desired result.\newline

One might make the case that linguistic differences are not that relevant when it comes to passwords, that the use of grammatical constructs is often trumped by patters of user behaviour in password creation that are international and well represented in rule-based password crackers. Thus, it should not matter where the network learns these patterns.

On the other hand, in our early attempts to train PassGAN on the Libero dataset we observed that most of the passwords the system generates attempt to mimic the sound and structure of Italian words; This leads us to speculate that the inclusion on natural language corpora might help the system to generate grammatically correct words, and thus perhaps improve performance.

Another interesting possibility could be the application of Transference Learning, which has proven useful in other cases such as \cite{Melicher2016}. By using Transference Learning to train on the Libero dataset and then on some natural language corpora, we might be able to nudge the system towards grammatically correct passwords without having to change the input data.\newline 
 
In conclusion, we believe this research might contribute some insight in the role that grammatical features have in password cracking, when this task is approached using Deep Neural Networks: many papers on the subject ask the question of whether Deep Learning systems are expressive enough to generate novel passwords and thus yield results that are not achievable with traditional password cracking methods, and we see this research as part of this ongoing quest into exploring the capabilities and limits of Deep Learning-based password crackers.

%\subsection{Problem formulation}
